\documentclass[de, pagenum]{fluxbeamer}
\usepackage{graphicx} % Required for inserting images
\usepackage{blindtext}
%\usepackage{lua-visual-debug}

\newcommand{\cbox}[1]{
\begin{tikzpicture}
        \filldraw [#1] (0, 0) -- (0, 2) -- (2, 2) -- (2,0) -- cycle;
    \end{tikzpicture}
}

\title[Beamer Vorlage]{LaTeX Beamer Vorlage der Flux Hochschulgruppe am KIT}
%\title[Beamer Vorlage]{a a a a a a a a a a a a a a a a a a a a a a a a a a a a a a a a a a a a a a}

\begin{document}
\titleframe
\begin{frame}{Diese Folie hat wirklich gottlos viel Text}
    \blindtext[1]
\end{frame}
\begin{frame}{Farben!}
    \begin{center}
        \begin{table}[]
            \begin{tabular}{cccccc}
                \cbox{flux-vio-dark }    &
                \cbox{flux-yellow-dark}  &
                \cbox{flux-blue-dark}    &
                \cbox{flux-orange-dark}  &
                \cbox{flux-green-dark}   &
                \cbox{flux-red-dark}       \\
                \cbox{flux-vio-light}    &
                \cbox{flux-yellow-light} &
                \cbox{flux-blue-light}   &
                \cbox{flux-orange-light} &
                \cbox{flux-green-light}  &
                \cbox{flux-red-light}
            \end{tabular}
        \end{table}
    \end{center}
    \begin{center}
        \begin{table}[]
            \begin{tabular}{cc}
                \cbox{flux-black}
                \cbox{flux-gray}
            \end{tabular}
        \end{table}
    \end{center}
\end{frame}
\end{document}
